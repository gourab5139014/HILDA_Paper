1. Gestural Query Specification

Data similarity between two attributes is calculated for a join operation. The number of times that each value from one attribute appears in the other is counted and a histogram is constructed from the counts for all of the values. This histogram is then converted to a multinomial distribution and its probability is measured under a dirichlet distribution.\\

2. Wrangler

Wrangler infers the data type of a column and highlights errors based on inconsistent data types.

Unfold creates new column headers from data values. Wrangler lets analyst extract data from a column and creates a new column, but the analyst has to name the new column manually.\\

3. Potter's wheel

Let's user define custom domains and structures data accordingly. The only function required to be implemented is an inclusion function match to identify values in the domain.

Unfold ”unflattens” tables; it takes two columns, collects rows that have the same values for all the other columns, and unfolds the two chosen columns. Values in one column are used as column names to align the values in the other column.\\

4. Yago

The model must be able to express entities, facts, relations
between facts and properties of relations.\\

5. Data Synthesizer

DataDescriber,
investigates the data types, correlations and distributions of the attributes
in the private dataset, and produces a data summary.

For each attribute, DataDescriber first detects whether it is numerical, and if so — whether it is an integer or a float. If the attribute is non-numerical, DataDescriber attempts to parse it as datetime. Any attribute that is neither numerical nor datetime is considered a string.