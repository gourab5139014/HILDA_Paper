% -*- root: ../paper.tex -*-

The \systemname knowledge-base is initially populated with a collection of example data sets.
This starting point produces low quality structure, requring careful curation of training data, as well as expert-provided heuristics.
This is intended to be an ongoing process, with continuous feedback from experts and users incrementally refining the knowledge-base.
In this section we outline the design of an interface that streamlines knowledge-base refinement, starting with a collection of random data
The central elements of this interface are 
(1) Visualizing the current quality of the knowledge base;
(2) Identifying problem name/match-quality function pairs;
(3) Refining records in the knowledge base by removing or merging existing data, or adding expert knowledge.


\begin{figure}
\begin{tabular}{r|l}
\textbf{How Y modifies X} & $(\namesymbol_x \oplus \namesymbol_y)(T_A)$ \\\hline
X Or Y & $max(\namesymbol_x(T_A), \namesymbol_y(T_A))$\\
X And Y & $\namesymbol_x(T_A) \cdot \namesymbol_y(T_A)$\\
X Unless Y & $min(\namesymbol_x(T_A), \namesymbol_y(T_A))$\\
Y Instead of X & $\namesymbol_y(T_A)$\\
Y Suggests X & $1-(1-\namesymbol_x(T_A))(1-\namesymbol_y(T_A))$
\end{tabular}
\caption{Example augmentation modifiers}
\label{fig:modifiers}
\end{figure}

\tinysection{Modifiers}
Expert knowledge in the knowledge-base is encoded in two parts: 
(1) A quality-match function that provides a heuristic encoding of the expert knowledge, and 
(2) An augmentation modifier that indicates how the new quality match function is to be combined with the existing one.  
Figure~\ref{fig:modifiers} illustrates several example modifiers together with intuitive phrasings of each.  
For example, if the expert heuristic defines an unrelated approach to matching columns, the highest match value is used.


\subsection{Common Refinement Challenges}

\subsubsection{Challenge 1: ...}