% -*- root: ../paper.tex -*-
We plan several extensions as future work focussed on building and refining a knowledge-base for storing column-naming heuristics. 

\tinysection{Use of contextual information} 
Contextual information such as units, data domains, and links between concepts and ontologies could be used to augment the \systemname knowledge-base. 
For example, we could use a network of semantic relations such as BabelNet strengthen data models for training the knowledge-base as well as providing curation recommendations to experts.~\cite{babelnet} 

\tinysection{Recommendation of columns for query} 
Concepts in the knowledge-base could be used to recommend columns that \textit{could be} in a query based on the columns that are already present.

\tinysection{\textit{Smarter} Matchers} 
We plan to develop matchers which regocognize a wider spectrum of data. For example, regular expression matchers could be used to detect geolocation data. Matchers which can identify \textit{synonymous} labels could help experts in the curation process. 

\tinysection{Other applications of context}
Once discovered, contextual information like units or data domains can be used in other ways.  
For example, if a column is identified as having a particular type of unit (e.g., `meters'), this knowledge could be used to warn users trying to merge it with an incompatible unit (e.g., `inches').  



  % Automatic translation/transformation (units, structure -- GPS vs Textual). Use regex.
  % \item Discovery of meta-data (e.g., units)