% -*- root: ../paper.tex -*-

The core component of \systemname is a knowledge-base that is used to identify column names.
In this section, we outline the key design challenges and capabilities of this knowledge-base.  
We first outline the two types of information being stored: Heuristics and Feedback.
Then we discuss both heuristics and feedback for one particular type of data: Numerical.  
Finally, we discuss support for both labeling and discovery queries over over the knowledge-base.

\subsection{Modeling Column Descriptions}

The \systemname knowledge-base is responsible for answering both labeling and discovery queries.
As previously discussed, both types of queries can be expresed in terms of a utility function $u(T[A], S[A])$, a $[0,1]$-valued measure of the quality of a match between a column name and a collection of data values in the column.
A utility of 0 indicates a non-match and a utility of 1 indicates a perfect match.  
We first discuss how this utility function is realized in the \systemname knowledge-base.

\begin{figure}
\placeholder{put bar-graph of types here}
\caption{Breakdown of data types}
\label{fig:type-breakdown}
\end{figure}

\tinysection{Heuristic Descriptions}
Fundamentally, the \systemname utility function decides whether a given description of a column name (i.e., $S[A]$) aligns with the content of the column (i.e., $T[A]$).  
The first challenge, then, is to precisely characterize how column names are described.
Specific description techniques vary by data type, so our first step was to prioritize based on available data.
We sampled a collection of \placeholder{\#\#\#} data sets from open data portals, as discussed later in Experiments (Section~\ref{sec:experiments}).
We then categorized the \placeholder{\#\#\#} columns in our sample into three broad types: 
(1) Numeric data, or any recods consisting of digits, at most a single decimal point, and an optional exponent; 
(2) Enumerated types, based on an arbitrary threshold of 100 distinct values in the column; and 
(3) Textual data, or anything else.  
By far, the dominant type was numeric, so the preliminary efforts we outline in this paper focus on describing numeric data.

\subsection{Numeric Data}
Focus on the challenges of sketching numeric types

Outline approximate matching rules: Expert heuristics, rules, etc...
\begin{itemize}
  \item How are approximate KB entries encoded?
  \item How do we unify different types of heuristics?
\end{itemize}


\tinysection{Feedback}

Outline exact matching rules: Data identity, recording/querying feedback.  Merging conflicts.

\begin{itemize}
  \item How is feedback saved in the KB.
  \item What happens in response to conflicting feedback.  
\end{itemize}



\subsection{Labeling Queries}

\subsection{Discovery Queries}
