% -*- root: ../paper.tex -*-

The core component of \systemname is a knowledge-base that is used to identify column names.
In this section, we outline the key design challenges and capabilities of this knowledge-base.  
We first outline the two types of information being stored: Heuristics and Feedback.
Then we discuss both heuristics and feedback for one particular type of data: Numerical.  
Finally, we discuss support for both labeling and discovery queries over over the knowledge-base.

\subsection{Modeling Column Descriptions}

The \systemname knowledge-base is responsible for answering both labeling and discovery queries.
For both we want to define a $[0,1]$-valued measure that we call the \emph{match-quality}: the quality of a match between a column name and a collection of data values in the column.
A match-quality of 0 indicates that the name is completely inappropriate for the column, while a match-quality of 1 indicates a perfect match.  
The \systemname knowledge-base is responsible for associating column names with match-quality functions for computing the descriptiveness of the name on a given column.
We will abuse syntax and use $\namesymbol$ to denote both the name, as well as the corresponding match-quality function ($\namesymbol : dom_T(A) \rightarrow [0,1]$) in the knowledge-base.

\begin{figure}
\placeholder{put bar-graph of types here}
\caption{Breakdown of data types}
\label{fig:type-breakdown}
\end{figure}


\subsection{Feedback}

Outline exact matching rules: Data identity, recording/querying feedback.  Merging conflicts.

\begin{itemize}
  \item How is feedback saved in the KB.
  \item What happens in response to conflicting feedback.  
\end{itemize}



\subsection{Labeling Queries}

\subsection{Discovery Queries}
