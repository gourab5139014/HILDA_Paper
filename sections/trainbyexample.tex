% -*- root: ../paper.tex -*-

To instantiate specific match-quality functions, we merge three sources of information: Learned Heuristics, Expert Augmentations, and User Feedback. 
The first category, learned heuristics, models the content and distribution of typical instances of columns with a similar name.
This distribution serves as a baseline match-quality function.
The second category, expert augmentations, modifies the first, increasing or decreasing values based on expert-provided descriptions of what should and should not appear in columns with this name.
The final category, user feedback, provides users with a way to confirm or override automated system choices, while also preserving these associations for future use.

\subsection{Learned Heuristics}
Learned heuristics serve as baseline match-quality functions by modeling typical instances of columns.
Specific modeling techniques vary by data type, so our first step was to assess what types exist.
We sampled a collection of 62 data sets from open data portals. 
%, as discussed later in Experiments (Section~\ref{sec:experiments}).

We then categorized the 458 columns in our sample into three broad types: 
(1) Numeric data, or any records consisting of digits, at most a single decimal point, and an optional exponent; 
(2) Enumerated types, based on an arbitrary threshold of atleast 50\% distinct values in the column; and 
(3) Textual data, or anything else.  
By far, the dominant type was numeric, so the preliminary efforts we outline in this paper focus on modeling numeric data.

\tinysection{Numeric Data}
We considered a range of options for modeling numeric data and settled on an approach based on numerical distributions.
In comparison to more complex approaches like neural networks, this approach is simple, efficient, and well understood.
Simply put, given a number of example column instances, we explore a range of numerical distributions and select the one with the best fit.
Our preliminary implementation of \systemname explores three different distributions: Uniform, Normal, and Log-normal. 

\begin{figure}
	\centering
	\includegraphics[trim={0 6mm 0 0},clip,width=1\columnwidth]{graphics/CDF_LexDistance}
	\caption{CDF of Lexicographical Distance for the best match for each column.}
	\label{fig:cdflexdist}
	\trimfigurespacing
\end{figure}

We used Kullback-Leibler (KL) divergence to measure how one distribution diverges from from another. A KL divergence of 0 suggess that distributions could be similar, if not the same. A KL divergence of 1 suggests that two distributions are very different. If knowledge of a specific problem domain suggests that similarity of data distributions could be better represented by another measure, KL divergence can be easily swapped with for it in our system.  

Lexicographical Distance indicates the different in the column labels of two distributions. We used two measures of Lexicographical Distance - Levenstein Distance and NGram Distance. Section~\ref{sec:expertui} talks about performance of both of the measures in estimating Lexicographical Distance on our dataset. A Lexicographical Distance value of 0 suggest that the two columns labels are exactly the same, whereas, a value of 1 suggests that the column labels are very different.

Learned heuristics require an assumption to build baseline match-quality functions. We proceed with the initial assumption that column pairs with \textit{similar} data would be labelled \textit{similarly}. The value of KL divergence of two columns is inversely proportional to \textit{similarity} of their data distributions. The value of Lexicographical distance of two columns is inversely proportional to \textit{similarity} of their column labels. We have verified this assumption by plotting the Cummulative Distribution Function of Lexicographical Distance with with minimum KL Divergence in Figure~\ref{fig:cdflexdist}. 80\% of column pairs with most similar data have their labels which are only 0.129 of less lexical distance apart. This indicates that columns pairs with the most similar data are indeed labelled similarly. 
 
For each distribution, we find the parameters ($\ell,h,\mu,\sigma,a$) that minimize the root-mean-squared (RMS) error between the theoritical and empirical distributions. 

$$\mathbb U(\ell, h)\;\;\;\;\;\mathbb N(\mu, \sigma)\;\;\;\;\;\mathbb Lognorm(\mu{*},\sigma{*})$$

Parameter estimation is performed by maximizing a log-likelihood function. Log-likelihood estimation is a widely used and robust technique for estimating the likelihood of a statistical distribution and its parameters being representative of the empirical data distribution. We used the SciPy package in Python. 

We then select the one distribution with the lowest overall RMS error. The resulting match-quality function is the probability of the column values being a representative sample drawn from this distribution.
% We measure this probability by the \placeholder{RMS Error} between the distribution of the sampled values and the distribution modeling the name.